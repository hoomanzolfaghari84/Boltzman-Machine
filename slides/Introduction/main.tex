% Use the `standard` option to get 4x3 aspect ratio.
% Use the `noul` option if there is a conflict with the `\ul` command.
\documentclass{antclass}
\newif\ifshow\showtrue

%\newcommand{\bf}{\mathbf}

%\pretitle{Version 2.2}
\title{Boltzman Machine}
\subtitle{Review and Prospects}
\author{Hooman Zolfaghari \\ Ramtin Moslem \\ Borna (lastname) \\ Houman (lastname)}

\begin{document}
\maketitle

\chapter{Energy Based Models}

\chapter{Botzman Machine}
\section{Introduction}
\begin{itemize}
 	\item Idea was taken from statistical physics and formulated by cognitive scientists.
 	
 	\item Energy function:
 	\[
 	E(x;\theta) = - (\bf{x}^T\bf{Wx}+\bf{b}^T\bf{x})
 	\]
 	\item Parameters same as in \textit{Hopfield} networks and \textit{Ising} models
 	
 
\end{itemize}

\section{Problem}

\begin{itemize}

	\item Hard to train (due to the partition function)
	\item Solution:
	\begin{itemize}
		\item Introducing latent variables
		\item  Restricting connections among observables.
	\end{itemize}

\end{itemize}

\section{Restricting BMs}
\begin{itemize}
	\item Consider: binary observable variables \(\bf{x}\in \{0,1\}^D \) and binary latent (hidden) variables \(\bf{z}\in \{0,1\}^M\)
	\item The relationships among variables
	specified through this \textit{energy function}:
	\begin{equation}
		E(\bf{x,z};\theta) = - \bf{x}^T\bf{Wz}-\bf{b}^T\bf{x} -\bf{c}^Tx
	\end{equation} 
	\item For this EF, the RBM is defined by the \textit{Gibbs} distribution:
	\begin{equation}
	p(\bf{x,z};\theta) = \frac{1}{Z_\theta}\exp(-E(\bf{x,z};\theta))
	\end{equation}
	
	
\end{itemize} 

\pagebreak

\begin{itemize}
	\item the \textit{partition function}:
	\begin{equation}
		Z_\theta = \sum_{\mathbf{x}} \sum_{\mathbf{z}} \exp \big(-E(\mathbf{x}, \mathbf{z}; \theta)\big)
	\end{equation}
	\item The marginal probability over observables (the likelihood of observation):
	\begin{equation}
		p(\mathbf{x}|\theta) = \frac{1}{Z_\theta} \exp \big(-F(\mathbf{x}; \theta)\big)
	\end{equation}
	\item where \(F(\cdot)\) is the \emph{free energy}:
	\begin{equation}
		F(\mathbf{x}; \theta) = -\mathbf{b}^\top \mathbf{x} - \sum_j \log \big(1 + \exp(c_j + (\mathbf{W}_{\cdot j})^\top \mathbf{x}) \big).
	\end{equation}
\end{itemize}

\section{RBM}
\begin{itemize}
	\item The presented model is called a restricted Boltzmann machine (RBM).
	\item  Useful property: the conditional distribution over the hidden variables factorizes given the observable variables and vice versa:
	
	\begin{align}
		p(z_m = 1 | \mathbf{x}, \theta) &= \mathrm{sigm}\big(c_m + (\mathbf{W}_{\cdot m})^\top \mathbf{x}\big),  \\
		p(x_d = 1 | \mathbf{z}, \theta) &= \mathrm{sigm}\big(b_d + \mathbf{W}_{d \cdot} \mathbf{z}\big). 
	\end{align}
	
\end{itemize}

\section{Learning RBMs}
\begin{itemize}
	\item For given data \(\mathcal{D}=\{\mathbf{x}_n\}_{n=1}^N\) \item Train an RBM using the maximum likelihood
	\begin{equation}
		\ell(\theta) = \frac{1}{N} \sum_{\mathbf{x}_n \in \mathcal{X}} \log p(\mathbf{x}_n \mid \theta)
	\end{equation}
	\item The gradient with respect to \(\theta\):
\begin{align}
	\nabla_\theta \ell(\theta) &= -\frac{1}{N} \sum_{n=1}^N \big( \nabla_\theta F(\mathbf{x}_n; \theta) - \sum_{\mathbf{\hat{x}}} p(\mathbf{\hat{x}}|\theta) \nabla_\theta F(\mathbf{\hat{x}}; \theta) \big)
\end{align}
	\item Cannot be computed analytically because the second term requires summing over all configurations of observables.

\end{itemize}
\pagebreak

\begin{itemize}
	\item One way to sidestep this: Stochastic approximation
	\item  Replacing the expectation under \( p(\mathbf{x}|\theta) \) by a sum over \( S \) samples \(\{\mathbf{\hat{x}}_1, \ldots, \mathbf{\hat{x}}_S\}\) drawn according to \( p(\mathbf{x}|\theta) \):
	\begin{align}
		\nabla_\theta \ell(\theta) &\approx -\frac{1}{N} \sum_{n=1}^N \nabla_\theta F(\mathbf{x}_n; \theta) - \frac{1}{S} \sum_{s=1}^S \nabla_\theta F(\mathbf{\hat{x}}_s; \theta).
	\end{align}
\end{itemize}

\pagebreak
\begin{itemize}
	\item A different approach: \emph{contrastive divergence}
	\item Approximates the expectation under \( p(\mathbf{x}|\theta) \) by a sum over samples \(\mathbf{\tilde{x}}_n\) drawn from a distribution obtained by applying \( K \) steps of the block Gibbs sampling procedure:
	\begin{align}
		\nabla_\theta \ell(\theta) &\approx -\frac{1}{N} \sum_{n=1}^N \big( \nabla_\theta F(\mathbf{x}_n; \theta) - \nabla_\theta F(\mathbf{\tilde{x}}_n; \theta) \big). 
	\end{align}
	 
\end{itemize}

%\pagebreak
%
%\begin{itemize}
%	\item \textbf{Step 1: Sample hidden units.}
%	\[
%	z_m^{(t+1)} \sim \mathrm{Bernoulli}(\mathrm{sigm}(c_m + \mathbf{W}_{\cdot m}^\top \mathbf{x}^{(t)})).
%	\]
%	\item \textbf{Step 2: Sample visible units.}
%	\[
%	x_d^{(t+1)} \sim \mathrm{Bernoulli}(\mathrm{sigm}(b_d + \mathbf{W}_{d \cdot} \mathbf{z}^{(t+1)})).
%	\]
%	
%	    \item \textbf{Positive Phase:} Compute \( \nabla_\theta F(\mathbf{x}_n; \theta) \) at data point \( \mathbf{x}_n \).
%	\item \textbf{Negative Phase:} Run Gibbs sampling for \( K \) steps to obtain \( \mathbf{\tilde{x}}_n \), then compute \( \nabla_\theta F(\mathbf{\tilde{x}}_n; \theta) \).
%	\item \textbf{Gradient Approximation:}
%	\[
%	\nabla_\theta \ell(\theta) \approx \frac{1}{N} \sum_{n=1}^N \big( \nabla_\theta F(\mathbf{x}_n; \theta) - \nabla_\theta F(\mathbf{\tilde{x}}_n; \theta) \big).
%	\]
%	
%\end{itemize}

\pagebreak

\begin{itemize}

	\item The original CD used K steps of the Gibbs chain, starting and is restarted after every parameter
	update.
	\item An alternative approach, Persistent Contrastive Divergence (PCD)
	does not restart the chain after each update
	typically resulting in a slower
	convergence rate but eventually better performance 
\end{itemize}

\section{Higher-Order Relationships }

\begin{itemize}
	\item The energy function allows the modeling of
	higher-order dependencies among variables.
	\item For instance: Third-order multiplicative interactions by introducing two kinds of hidden	variables
	\begin{itemize}
		\item \textbf{Subspace Units:} Reflect feature variations, robust to invariances:
		\item \textbf{Gate Units:} Activate subspace units, pool subspace features.
	\end{itemize}
\end{itemize}

\pagebreak
\subsection{Random Variables for SubspaceRBM}
\begin{itemize}
	\item Observables: \( \mathbf{x} \in \{0, 1\}^D \).
	\item Gate Units: \( \mathbf{h} \in \{0, 1\}^M \).
	\item Subspace Units: \( \mathbf{S} \in \{0, 1\}^{M \times K} \).
	\item \textbf{Connections:} \( x_i \), \( h_j \), and \( s_{jk} \).
\end{itemize}

\pagebreak

\subsection{Energy Function for SubspaceRBM}
	\begin{align}
	E(\mathbf{x}, \mathbf{h}, \mathbf{S}; \theta) =
	-\sum_{i=1}^D \sum_{j=1}^M \sum_{k=1}^K W_{ijk} x_i h_j s_{jk} \nonumber \\
	-\sum_{i=1}^D b_i x_i 
	-\sum_{j=1}^M c_j h_j 
	-\sum_{j=1}^M h_j \sum_{k=1}^K D_{jk} s_{jk},
	\end{align}
	\[
	\theta = \{W, \mathbf{b}, \mathbf{c}, \mathbf{D}\}, \quad 
	W \in \mathbb{R}^{D \times M \times K}, \quad \mathbf{b} \in \mathbb{R}^D, \quad \mathbf{c} \in \mathbb{R}^M, \quad \mathbf{D} \in \mathbb{R}^{M \times K}.
	\]
	
	
\pagebreak
\subsection{Conditional Distributions in SubspaceRBM}
\begin{gather}	
	p(x_i = 1 | \mathbf{h}, \mathbf{S}) = \mathrm{sigm}\left(\sum_j \sum_k W_{ijk} h_j s_{jk} + b_i\right) \\
	p(s_{jk} = 1 | \mathbf{x}, h_j) = \mathrm{sigm}\left(\sum_i W_{ijk} x_i h_j + h_j D_{jk}\right) \\
	p(h_j = 1 | \mathbf{x}) = \mathrm{sigm}\left(-K \log 2 + c_j + \sum_{k=1}^K \mathrm{softplus}\left(\sum_i W_{ijk} x_i + D_{jk}\right)\right)
\end{gather}



\chapter{Research Prospect}

\chapter{Template Things}
\section{Single Column}

For 4-by-3 aspect ratio slides, specify \verb|standard| as an option to the
document class. Write your presentation like a normal \LaTeX\ file with a
\verb|\maketitle| command and \verb|\chapter| and \verb|\section| headings. The
\verb|\maketitle| contents are defined by the following macros:
\begin{center}
    \begin{tabular}{l@{\qquad}l}
        \verb|\pretitle| &
        \verb|\author| \\
        \verb|\title| &
        \verb|\subtitle|
    \end{tabular}
\end{center}
The \verb|\chapter| heading creates a slide with just the chapter name, and the
\verb|\section| heading sets the title of a new slide. However, if no text
follows the section, no slide will be created. Text which does not fit on one
slide will flow onto the next slide automatically.

\section{Double Column}\twocolumn\raggedright

Use the \verb|\twocolumn| and \verb|\onecolumn| commands right after the section
heading to control the number of columns. Text will flow from the left column to
the right.
\begin{itemize}
    \item Point one
    \item Point two
    \item Point three
    \item Point four
    \item Point five
    \item Point six
    \item Point seven
    \item Point eight
    \item Point nine
    \item Point ten
    \item Point eleven
    \item Point twelve
\end{itemize}
You can use \verb|\pagebreak| to force text onto the next column.

\section{Table of Stuff}

You can create any variety of subdivisions on your slide by using the
\verb|tabular| environment.
\begin{center}
\begin{tabular}{C{0.25\textwidth}cC{0.25\textwidth}cC{0.25\textwidth}}
    \cellcolor{colorH}\textcolor{white}{Primary}\strut &&
    \cellcolor{colorH}\textcolor{white}{Secondary}\strut &&
    \cellcolor{colorH}\textcolor{white}{Tertiary}\strut \\
    First && Second && Third \\
    One && Two && Three \\[1em]
    \cellcolor{colorH}\textcolor{white}{Alpha}\strut &&
    \cellcolor{colorH}\textcolor{white}{Beta}\strut &&
    \cellcolor{colorH}\textcolor{white}{Gamma}\strut \\
    Green && Blue && Red \\
    Cyan && Yellow && Magenta
\end{tabular}
\end{center}
The \verb|\cellcolor| command sets the background color of a table cell.

\section{Centering}

\begin{Center}
    \Large Use the \texttt{Center} environment \\
    to center horizontally \emph{and} vertically.
\end{Center}

\chapter{Explicit Code}

\section{Python}\onecolumn

\ifshow
Use the \verb|python| environment for Python code.
\begin{python}
def write_list(fid, x, level):
    ind = '  '*level
    xs = '0' if abs(x[0]) < 1e-3 else "%.3f"
    txt = '\n%svalues=\"%s' % (ind, xs)
    for n in range(1, len(x)):
        xs = '0' if abs(x[n]) < 1e-3 else "%.3f"
        if len(txt) + 3 + len(xs) >= 80:
            fid.write(txt + ';\n')
            txt = ind + '  ' + xs
        else:
            txt += '; ' + xs
    fid.write(txt + '\"')
\end{python}
\pagebreak\addtocounter{page}{-1}
\fi
You can use the \verb|`\HL`| command to highlight a line of code.
\begin{python}
def write_list(fid, x, level):
    ind = '  '*level
    xs = '0' if abs(x[0]) < 1e-3 else "%.3f"
    txt = '\n%svalues=\"%s' % (ind, xs)
    for n in range(1, len(x)):
        xs = '0' if abs(x[n]) < 1e-3 else "%.3f"
        if len(txt) + 3 + len(xs) >= 80:
`\HL`            fid.write(txt + ';\n')
            txt = ind + '  ' + xs
        else:
            txt += '; ' + xs
    fid.write(txt + '\"')
\end{python}

\section{MATLAB}

Use the \verb|matlab| environment for MATLAB code.
\begin{matlab}
function savepdf(name, width, height)
    % name is the file name including ".pdf".
    % Both width and height are in (cm).
    set(gcf, 'units', 'centimeters', ...
        'position', [0, 0, width, height])
    set(gca, 'FontSize', 9);
    set(gca, 'FontName', 'Times New Roman');
    exportgraphics(gcf, name, ...
        'ContentType', 'vector');
end
\end{matlab}

\section{R Language}

Use the \verb|rlang| environment for R code.
\begin{rlang}
factorial <- function(n) {
    if (n == 0 || n == 1) {
        return(1)
    } else {
        return(n * factorial(n - 1))
    }
}
\end{rlang}

% You can create blank pages for placing full-page graphics or text with the
% `\blankpage` command.
\blankpage
% You can place images and text at arbitrary locations with the `\pos` command.
\pos{0pt}{0pt}{\tikz{
    \newcounter{density}\setcounter{density}{10}
    \def\mainColor{colorH}
    \path[coordinate] (0,0) coordinate(A)
        ++(\paperwidth,0) coordinate(B)
        ++(0,-\paperheight) coordinate(C)
        ++(-\paperwidth,0) coordinate(D);
    \fill[\mainColor!\thedensity]
        (A) -- (B) -- (C) -- (D) -- cycle;
    \foreach \x in {1,...,18}{%
        \pgfmathsetcounter{density}{\thedensity+5}
        \setcounter{density}{\thedensity}
        \path[coordinate] coordinate(X) at (A){};
        \path[coordinate] (A)
            -- (B) coordinate[pos=0.15](A)
            -- (C) coordinate[pos=0.15](B)
            -- (D) coordinate[pos=0.15](C)
            -- (X) coordinate[pos=0.15](D);
        \draw[\mainColor!80!black, fill=\mainColor!\thedensity]
            (A) -- (B) -- (C) -- (D) -- cycle;
    }
}}
\pos[6cm]{0.125\textwidth}{2cm}{
    \raggedright\large Blank pages with the ``\texttt{\textbackslash
    blankpage}'' command.
}
\pos[6cm]{0.6\textwidth}{6cm}{
    \raggedright\large Arbitrary positioning with the ``\texttt{\textbackslash
    pos}'' command.
}

\closing
\end{document}
